% -----------------------------------------------------------------------------
%                                 Related Work
% -----------------------------------------------------------------------------
\newpage                                                 \chapter{Related Work}



A binaural interface would consist of two main elements: a pseudo intelligent
agent able to classify the importance of specific notifications, and the
binaural engine.  The following section explores research that have explored
both components as well as demontstrating the interplay between the different
elements of this interface.



When designing an interface, there should be three main considerations.
How can you make it usable ?
Who is the user ?
What is the task?



%-----------------------------------------------------------------------------%
\section{                  Psychology of Interface Design                     }

\subsection{                  Mental Models and Cognitive Loads               }

The ultimate goal User Interface Design is to mediate effective operation and
control of a machine from the operator's perspective. The interplay of
ergonomics and psychology are known facets of the field. As such, the design
of novel interfaces are encouraged to either increase efficiency or user
satisfaction (preferably both).

To more concretely enumerate the goals, the literature provides a number of
guiding examples.  Exploring a number of different goal oriented tasks, general
patterns of interactions emerge that suggest the style that Irys should present.


A limitation of most interfaces is simply the size and volatility of a user's
short term memory.  To make real time captioning easier for untrained works
Lasecki et al. demonstrated how time could be warped for any given users task.
Warping the playback speed of previously recorded audio enabled users to perform
the captioning task by reducing the high cognitive load needed for captioning
~\cite{lasecki2013warping}.

Simplifying information is not only beneficial for goal oriented and task
driven processes, but they are also emperically preffered by users of a
number of systems.

Exploring a users collective preference for simpler representations of
information. For example, travelling users preferred summarized directions that incorporated familiar paths to more
detailed and exact routing instructions.  The users in the study actually
valued travelling on familiar routes enough to sacrifice overall trip distance
~\cite{patel2006personalizing}.



When an interface is controlled by a semi-autonomous intelligent agent, it
becomes increasingly important to be aware of the interactions that users will
have with the system as decisions are being made.  In this system, the
decisions that would be made relate to the verbosity for a given notification,
the position to place the notification relative to the user, and what actions
are available to the user when acting on a notification.  Prior work has
explored how an intelligent agent can explain itself to the user, the amount
varying degrees by which an intelligent agent can reconfigure itself, and the
effects of a user's mental model soundness~\cite{kulesza2012tell}.






%-----------------------------------------------------------------------------%
\section{                  Immersive Interfaces                               }

Kayur Patel defined a fully immersive virtual reality setting to be an
extension of virtual reality that also captures a user's full body motion
while immersing the subject into a virtual environment.  The exploration of
such an environment for educational purposes allowed researchers at UC Berkeley
to conclude that the increased interactions led to more effective
education~\cite{patel2006effects}.






%-----------------------------------------------------------------------------%
\section{                  Interface Elements                                 }
\subsection{                  Visual Notifications                            }

Given the stochastic nature of notifications, probabilistic reliability can be
used to create intelligent messaging sytems that gain a user's trust, rather
than lose it~\cite{leetiernan2001effective}. Emperical evidence show that users
will ignore all notifications that are not highly valid when performing
demanding visual tasks~\cite{maltz2000cue}.

The question of a users trust in a system is therefore important when designing
notification interfaces, with the goal being mitigating all negative first
impressions.  Empirical evidence shows that users carry a historical bias when
dealing with actionable notification \cite{leetiernan2001effective}.

The study of information design and options that are suitable for three often
conflicting design objectives of notifications - interruption to primary tasks,
reactions to specific notificaitons, and comprehension of information over time
- are necessary.






%-----------------------------------------------------------------------------%
\section{                  Tactile Notifications                              }






%-----------------------------------------------------------------------------%
\section{                  Auditory Notifications                             }

SOCIAL ISSUES: In their paper Hanson et al. discussed the interplay of social
situations and auditory cues~\cite{hansson2001subtle} Current auditory
notifications cues can be attention demanding, distinct, and can be percieved as
intrusive in social situations. "The beeping and ringing is by nature an
intrusive sound not unlike the sound of an alarm clock" referring auditory cues
often heard arising from cell phones  ~\cite{hansson2001subtle}.






%-----------------------------------------------------------------------------%
\section{                  Current Audio Based Accessibility Solutions        }






%-----------------------------------------------------------------------------%
\section{                  Interface Design                                   }






%-----------------------------------------------------------------------------%
\section{                  Interfaces                                         }






%-----------------------------------------------------------------------------%
\section{                  Engines                                            }








%-----------------------------------------------------------------------------%
\section{                  Analytics                                          }
\subsection{                  Design Principles                               }
\subsection{                  Design Patterns                                 }
\subsection{                  Design Methodologies                            }
\section{                  Artificial Intelligence                            }
