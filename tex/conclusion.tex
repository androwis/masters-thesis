% -----------------------------------------------------------------------------
%                                 Conclusions
%-----------------------------------------------------------------------------
\newpage                                                  \chapter{Conclusions}

The lack of adoption of 3D audio as a primary interface is often attributed to a
number of factors in the literature. Prior research in psychology has shown that
humans base much of their communication on gestures, nuance, and inflection
~\cite{thackara2005bubble}. As a result of modern speech synthesizers’ inability
to communicate using these components, existing systems are often perceived as
ineffective or poor communicators. Audible interfaces that are not based purely
on speech, but complement other sensation modals, focusing on other kinds of
non-communication based sounds have experienced more promising results (as is
often demonstrated with games and movies)~\cite{thackara2005bubble}.

When using sound as a communication medium to interact with humans, certain
factors need to be considered due to humans’ sensitivity to sound. An interface
designed around audio must understand the psychological basis for tone, nuance,
and inflections when portraying information to a subject. Humans have no choice
but to follow an auditory patterning as long as it does not consist of too much
distortion in the sense of noise pollution. Human’s inability to ignore most
sound plays an important role when creating an interface based primarily on
sound to avoid user frustration.

We have explored these psychological effects and explored the background
research needed to create four different patterns of interfaces.  Much of our
society is driven by information. Armed with devices that are constantly
connected, the current generation of technology has the potential to communicate
massive amounts of information, everything from weather forecasts and traffic
conditions, to neighboring attractions, restaurant schedules, store specials,
even to the location and discoveries of our friends. Fields of research have
explored how to best communicate constantly changing information to interested
parties at the appropriate time.  With the influx of mobile devices that provide
an always on channel, research has explored the effect disruptions have on a
multitasking computing environment. The goal of much of this research has been
to study how relevant and correct information can be efficiently delivered to a
user in a manner that does not distract from their current
tasks\cite{McCrickard2003509}.

We have presented an exploration of prior research and posed necessary research
questions that should be explored when designing an interface that uses 3D audio
to place sound around a user on any device. Such an interface could leverage
techniques in binaural audio, artificial intelligence, and human computer 
interaction to provide users with an immersive environment to interact with 
their technology. This interaction would be useful, both as a tool
for enabling blind users, augmenting the capabilities of non-handicapped
individuals, and as an approach to test spatial layout of information 
for humans when interacting with machines.


% -----------------------------------------------------------------------------
%                                     fin
% -----------------------------------------------------------------------------
\newpage                                                  \bibliography{tex/area}{}
\bibliographystyle{plain}
\end{document}
