% -----------------------------------------------------------------------------
%                                 Future Work
% -----------------------------------------------------------------------------
\newpage                                                  \chapter{Future Work}

3D audio interfaces present a number of exciting capabilities in human computer
interaction. With an initially completed framework and literature review, the
immediate short-term goals for this project are to perform user studies and
performance measures on the efficacy of this type of interface as it relates to
specific tasks.

The targeted user base for this interface are members of the blind or low vision
community. We plan on performing more in-depth studies of how this interface can
be used to best enable blind users to interact with a given interface through
multi-tasking techniques afforded by independent sound objects.

%-----------------------------------------------------------------------------%
\section{                  Empirical Measure of Efficiency                    }

Having multiple sources of audio may be distracting for a user, so evaluation on
the number of voices a user can focus on, what types of information are best
presented to the user, and time locality are all other metrics of interest for
this interface.   Search and navigation within this type of interface becomes an
interesting research topic.  How can a user query information audibly.
Systems, such as Apple’s Siri and Google Voice attempt to provide an interface
for general query and answer interactions, but how can a system be built to
allow for in-depth querying of content in a spatial manner?  Should context be
provided to search or should the system only search the locality around the
user?

Most importantly, the next major focus will be on quantifying the benefits of
this type of interface. Metrics on goal completion on tasks in a 3D space as
compared to regular interfaces as well as throughput as measured by multi-task
capacity would assist in understanding the efficacy of this system. Finally,
we’re very excited to explore the ability of 3D audio in helping users remember
information by providing a tangible dimension to their information processing.

Human computer interaction has very well established guidelines and methods for
evaluating graphic interfaces. There are over 13 commonly accepted design
principles encompassing many facets of visual perception and information
processing~\cite{wickens2004intro}. Expanding on this research in
the auditory domain with empirical studies on user satisfaction using these
measures is the first step in understanding the potential for 3D interfaces.


%-----------------------------------------------------------------------------%
\section{                  Artificial Intelligence Domain                     }

Before exploring the contributions of such an interface to the field of human
and computer interaction, there are a number of technical implementations that
would contribute to artificial intelligence.  Prior literature has
demonstrated that acute attenuations are mis-calibrated in a user's head
related transfer function, then the auditory experience degrades sharply
~\cite{algazi2001cipic}.  One contribution to this domain would be a systemic
learning procedure that is able to model the parameters needed for an
individual's head related transfer function.  The CIPIC HRTF database provides
high resolution anthropometric measurements and will act as a strong source of
data for the parameter tuning for such a classifier.  Manual Techniques for
calibrating HRTF were presented by researchers from Creative Audio and would
provide a great stepping stone for this ~\cite{jost2000transaural}.


%-----------------------------------------------------------------------------%
\section{                  Crowd Sourced Domain                               }

Crowd sourced work has proven to be an effective medium for solving hard AI
problems, computer vision and other computationally ambiguous problems.  With a
binaural interface, a new control mechanism could be created that would allow
other users to provide navigational cues where a full audible stream may not
be beneficial.  Examples that demonstrate how to utilize audio as a control have
been the dropping of digital soundtracks in the augmented reality space.  Other
uses for this could be orchestrated guidance or remote control of a user to a
location (such as navigating a blind user to a bathroom in a mall).  Another
application could be providing more natural instruction to web workers through
the use of sound images that guide workers to complete tasks.



%-----------------------------------------------------------------------------%
\section{                  Analytics Domain                                   }

Researchers have shown that the use of analytics such as eye-tracking for
visually enabled users looking at search data provided rich insights into the
user's intent, the page's efficiencies, and the user's goals when browsing
search results.  By looking at the time the eye focuses on a given web page,
they were able to extract rich features impacting areas such as usability,
interface design, and other valuable aspects~\cite{granka2004eye}.  To this we
ask whether the same insights can be gained by observing how a non-sighted user
interacts with content.  The implementation of the particle filtering algorithm
to infer location is currently the set as the first step for this project.



%-----------------------------------------------------------------------------%
\section{                  Cultural Domain                                    }

How does culture, age, or language affect an individual's choice or mode of
communication?  As this interface is primarily targeted to providing auditory
speech-like interactions between a number of running processes and applications
to a user, communication preferences begin to play an important role in
determining optimal configuration.  Surveys have shown that culture influences
on communication preferences for patients seeking treatment at the end of life
stage present different requirements in both the content and structure of the
information relating to care~\cite{shrank2005focus}. This cultural sensitivity
to content can be explored to tailor this interface to users with diverse
backgrounds.
