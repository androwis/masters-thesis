% -----------------------------------------------------------------------------
%                                 Future Work
% -----------------------------------------------------------------------------
\newpage                                                  \chapter{Future Work}

3D audio interfaces present a number of exciting capabilities in human computer
interaction. With an initially completed framework and literature review, the
suggested short-term goals for this project are to perform user studies and
performance measures on the efficacy of this type of interface as it relates to
specific tasks.

Because there are two targeted user bases for this interface, the empirical
measures are inherently different. The first group consists of members of the
blind or low vision community, and the other is comprised of non-disabled users
performing other tasks that monopolize their vision (such as driving or
walking).

The first set of evaluation goals are to perform more in-depth studies of
how this interface can be used to best enable each group of users to interact
with a given interface through multi-tasking techniques as afforded by the 3D
placement of independent sound objects.

%-----------------------------------------------------------------------------%
\section{                 Empirical Measure of Efficiency                    }

Having multiple sources of audio may be distracting for a user, so evaluation on
the number of voices a user can focus on, what types of information are best
presented to the user, and the optimal time locality of notifications are all
metrics of interest for this interface. Search and navigation within this type
of interface becomes an interesting research topic. How can a user query
information audibly. Systems, such as Apple’s Siri and Google Voice attempt
to provide an interface for general query and answer interactions, but how can
a system be built to allow for in-depth querying of content in a spatial manner?
Should context be provided to search or should the system only search the
locality around the user?

Formally, we are interested in three aspects of binaural interfaces which can be
used to quantify their efficiency for the target users.

%-----------------------------------------------------------------------------%
\subsection{                  Proposed Next Steps                             }

Most importantly, the next major focus should be on quantifying the benefits of
this type of interface. Metrics on goal completion on tasks in a 3D space as
compared to regular interfaces as well as throughput as measured by multi-task
capacity would assist in understanding the efficacy of this system. Finally,
researching the ability of 3D audio to help users remember information by
providing a tangible dimension to their information processing can lead to
insights on the benefits these types of systems can have for their user bases.

The most important question we pose looks at the ability of these interfaces to
reduce the users cognitive load. We describe auditory interfaces whose
conceptual model maps interface elements into a 3D audio space. The hypothesis
is that the extra dimension in the audio can reduce the users cognitive load. As
binaural interfaces are built, one goal is to explore the features that allow
audio and directionality to reduce the cognitive load for the user. As these 3D
interfaces provide information to the user in terms of spatial attenuation and
audio feedback, we can begin to explore the added benefits these new features
have on improving both the users comprehension of content presented to them and
their efficiency with interacting with the systems as they provide users with
cues to assist recall of information.

We then shift our focus to the role of audio in interface design. Because of
sound's transient properties, its ability to portray or navigate through
interfaces elements changes. More importantly, audible interfaces provide an
alternate mode of access to computers whenever a users’ visual focus is
unavailable. Visual focus can be unavailable for a number of reasons, such as
when users are driving, using hand-held devices while engaged in other physical
activities that command visual attention, or even as a result of physical
disabilities~\cite{ michelis2008disappearing}. We are interested in
understanding how a user's interaction with sound can quantitatively be used to
more efficiently accomplish tasks in different scenarios.

Coupling 3D audio interfaces with intelligent agents driving notifications
presents the next question. Future work should quantify the efficiency
different types of intelligent agent models have when coupled with auditory
interfaces as measured by a user's ability to multi-task, context switch,
navigate through their content.

Finally, we will explore the decision making processes within each scope of
intelligent agent and audio scheme. With each type of stimuli or notification,
there is a stochastic decision that must be made regarding both the best manner
and the proper time to alert the user. Future work can begin to  explore how
these trade offs present themselves in the auditory domain. Human computer
interaction has very well established guidelines and methods for evaluating
graphic interfaces. There are over 13 commonly accepted design principles
encompassing many facets of visual perception and information
processing~\cite{wickens2004intro}. Expanding on this research in the auditory
domain with empirical studies on user satisfaction using these measures is the
first step in understanding the potential for 3D interfaces.
